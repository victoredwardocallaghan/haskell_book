%
% Copyright © 2013 Edward O'Callaghan. All Rights Reserved.
%

\chapter*{Preface}
\label{cha:preface}

\addcontentsline{toc}{chapter}{Preface}

\subsection*{Book development}
The following people participated in the development of this text.

\begin{multicols}{2}{
\begin{itemize}
\item[] Edward O'Callaghan
\end{itemize}
}
\end{multicols}


\subsection*{About this book}


This book had its genesis from lecture notes I compiled myself while at
university. I wrote many notes as I felt things were ether too fragmented,
disordered or just poorly explained. With the modern advances in mathematics,
type theory in particular, we are on the precious of a new era in computer
science, where computer science will be given back to mathematicians. A merger
of disciplines on seemly two different trajectories into a more unified
discipline of rigours computational linguistics.

However, computer science and mathematics are separately taught fields and so,
as a result, has lead to this artificial barrier.This systemic problem has
been further aggravated by computer science actively avoiding the underlaying
mathematics and mathematics paying little to no interest in undergraduate
computational theory. A number of topics that have a deep unexpected connection
that are prevalent in both disciplines are covered. In particular,
\emph{dependent types}, \emph{category theory} and \emph{homotopy type theory}
are some of the quintessential topics of interest. This text serves to
introduce the reader to programming and the foundations of mathematics from
scratch by way of \emph{homotopy type theory} and \emph{category theory} and
explores aspects of these topics to lead to a new era in high assurance
software design and implementations by programs as proofs. We complement this
with the concept of propositions as types as finally consider the connection
between \emph{category theory} and \emph{homotopy type theory}. We also look
at the theory of computation by looking at \emph{regular languages},
\emph{minimal automata}, \emph{syntactic monoids}, \emph{Turing machines} and
\emph{decidability} to number but a few topics. The exposition is gentle
however rigorous in the hopes to provide a proper foundation for later study
and continued research.

This text serves my interest to once more marry these fields and provide a
standard introductory text for undergraduates. My hope is to make this text
accessible as to serve to support development of a modern rigours treatment
from first to third year undergraduate.

%TODO explain content..
%In chapter 1 ..


\bigskip

\begin{flushright}
Foundations of Type\\
Edward O'Callaghan\\
Australia, Jun 2013
\end{flushright}
