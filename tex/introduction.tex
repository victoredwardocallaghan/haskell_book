%
% Copyright © 2013 Edward O'Callaghan. All Rights Reserved.
%

\chapter*{Introduction}
\markboth{\textsc{Introduction}}{}
\addcontentsline{toc}{chapter}{Introduction}
\setcounter{page}{1}
\pagenumbering{arabic}

In this text we explore the essential material required in the pursuit of praxis
in the concept ideas of \emph{computational trinitarianism}.

We consider notions such as;
\begin{itemize}
  \item propositions as types,
  \item programs as proofs,
  \item and relations between type and category theory.
\end{itemize}

This deep unexpected connection is referred to as
“computational trinitarianism” %FIXME?? ~\cite{Harper}

\begin{verse}
The central dogma of computational trinitarianism holds that
Logic, Languages, and Categories are but three manifestations of one divine
notion of computation. There is no preferred route to enlightenment: each
aspect provides insights that comprise the experience of computation in our
lives.

Computational trinitarianism entails that any concept arising in one aspect
should have meaning from the perspective of the other two. If you arrive at an
insight that has importance for logic, languages, and categories, then you may
feel sure that you have elucidated an essential concept of computation – you have
made an enduring scientific discovery.
\end{verse}

..
foo \index{principle}
bar \label{defn:defeq}
